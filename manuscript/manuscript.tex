
\documentclass[a4paper,twoside]{article}
\usepackage{booktabs,multirow,bigstrut,rotating,graphicx}
\usepackage[utf8]{inputenc}
\usepackage{natbib}
\usepackage{placeins}
\usepackage{amsmath}
\usepackage{hyperref}
\hypersetup{colorlinks,breaklinks,
            linkcolor=blue,urlcolor=blue,
            anchorcolor=blue,citecolor=blue}
\usepackage{subfig}
\usepackage{longtable}
\usepackage{lscape}
\usepackage{breqn}
\usepackage{enumitem}
\usepackage{cleveref}
\usepackage{microtype}
\usepackage{colortbl}
\usepackage{bm}
\usepackage[official]{eurosym}

\newcommand{\note}[1]{\textcolor{red}{[{\bf #1}]}}

\bibpunct{(}{)}{,}{a}{}{,} 

% =========================== 
% IMPORTANT NOTE, PLEASE READ 
% =========================== 
%
% This template helps you writing an article in LaTeX for REGION.
% Since we do the layout editing in LaTeX, we highly recommend that
% you already write the article in LaTeX. BUT, you should follow
% some simple instructions and NOT try to apply the full set of
% functions of LaTeX. This will save you and us a lot of work because
% otherwise we will have to undo later all the fancy coding and packages
% that you have entered.
%
% MOST IMPORTANT:
%1. PLEASE, DO NOT use any other packages than the ones defined above.
%   The packages above have been tested to work in our setup and to produce
%   our "house style" output. Other packages may be incompatible or screw
%   up the formatting. 
%2. DO NOT invest much time into formatting of the text. In particular, 
%   do not explicitly insert horizontal or vertical space. It is the task
%   of the layout editor to make the paper look good (and according to the
%   "house style"). Your task as the author is only to submit readable text.

% Everything above this line should remain unchanged!!!

% The next five lines define the space for text on the page. These lines are copied from REGION's
% regart-style to ensure that formatting is close to the final one even when one uses the standard
% article style (see Introduction below).

\setlength{\textwidth}{13.5cm}
\setlength{\textheight}{24.3cm}
\setlength{\oddsidemargin}{1.3cm}
\setlength{\evensidemargin}{1.3cm}
\setlength{\topmargin}{-1cm}


% The next few lines (down to "\begin{document}") define the title information.
% Please, substitute the values for \title and \author with the text appropriate for your
% submission. The commented lines will be used in layout editing. Please, leave
% leave them commented and do not change them.

\title{A geocomputational notebook to monitor regional development: The case of Bolivia}


\author{}

\date{} % Leave this empty to avoid any date being printed

%\setcounter{page}{1} % CHANGE!!
%\renewcommand{\thepage}{\arabic{page}}
%\jvol{2} % \jvol{1}
%\jnum{2} % \jnum{1}
%\jyear{2015} % \jyear{2014}
%\jpages{1--26} % \jpages{1--3}
%\jauthor{} % P.\ Lagas, F.\ van Dongen, F.\ van Rijn, H.\ Visser

%\received{} % \received{9 April 2014}
%\accepted{} % \accepted{9 April 2014}


\begin{document}

% Here starts the section where your text will go. Please, consult
% the extended template (template.tex) for examples.

\maketitle

\begin{abstract}
% Your abstract goes here
\end{abstract}

\section{Introduction}



\section{Cloud-based geocomputational environment}

Introduce Deepnote

\section{New database to monitor regional development}



\section{Methods for exploratory spatial data analysis}

Introduce 3 ESDA methods

\subsection{Mapping the spatial distribution of development}

\subsection{Spatial dependence: Global clustering and local clusters}

\subsection{Spatial heterogeneity: Geographically weighted regression}


\section{Results: Regional development patterns of Bolivia}

\subsection{Mapping the spatial distribution of development}

\subsection{Spatial dependence: Global clustering and local clusters}

\subsection{Spatial heterogeneity: Geographically weighted regression}




\section{Concluding remarks}


\end{document}
